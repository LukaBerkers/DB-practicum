\documentclass[a4paper, dutch, parskip=full]{scrartcl}
\usepackage[dutch]{babel}

\begin{document}

\title{Postnormalisatie}
\author{Luuk Berkers & Nathan van den Berg}
\date{\today}
\maketitle
Bij het normaliseringsproces merkten we slechts 4 FDs op.
Gegeven een land en een spel dan is er \'{e}\'{e}n unieke patenthouder.
Verder wordt de prijs van een versie van een spel volledig bepaald door de verkoper en die versie zelf.
Het is immers niet logisch om twee dezelfde producten in \'{e}\'{e}n winkel voor twee verschillende prijzen te verkopen.
In beide gevallen zijn deze tabellen al in de verst gereduceerde vorm, dus opsplitsen bleek niet nodig.
Voor de winkels geldt dat adress een ander key is voor een offline winkel en een website url is een andere key voor een webshop (naast de gegeven key: `sid').
Maar omdat url's en adressen uniek zijn, zijn ze superkeys, dus de twee tabellen zijn in BCNF.
Omdat er verder geen niet-triviale FDs zijn is het relationele schema in BCNF.
We zien ook dat het schema DP zijn. \\

Verder zijn er nog kleine aanpassingen gemaakt aan de directe conversie van ER diagram naar een relationeel schema.
Zo hebben we een 'Job' tabel togevoegd in plaats van twee `Art' en `Design' tabellen.
Dit hebben we gedaan, omdat we zo een tabel uitsparen en het qua redundancy niet veel uitmaakt.
Ook kunnen we later het aantal banen vrij makkelijk uitbreiden.
De ISA structuur aan het linkerkant van het diagram, hebben we versimpeld tot twee tabellen: `OnlineWebshop' en `OfflineWebshop'.
Hierbij hebben we geforceerd dat de id's van de winkels niet overlappen.
Dit is handig voor queries later als we de tabellen willen verenigen.
\end{document}
